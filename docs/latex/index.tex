\subsection*{Brief}

The Signal\&Power Management Board ({\bfseries \hyperlink{namespaceSPMB}{S\+P\+MB}}) provides firmware for Arduino U\+NO devices using Atmega328p microcontrollers facilitating a Robot Operating System ({\bfseries R\+OS}) interface for a radio controlled ({\bfseries RC}) supervised vehicle. Here the current configuration is based on the {\bfseries T\+R\+A\+X\+X\+AS} platforms (in specific T\+R\+X4). The \hyperlink{namespaceSPMB}{S\+P\+MB} provides direct RC forwarding and publishing of actuated control signals to the R\+OS interface. With a predefined fast switching sequence of channels the software based control algorithms are activated. Furthermore a failsafe for idled software connections is embedded such that RC operation is activated if less than {\itshape 10\+Hz} control frequency is present. If a RC channel is not read successfully, e.\+g. broken wires or empty batteries, the channel is deactivated and the output goes in the channels default control signal. In case of velocity and steering this corresponds to 0 actuation. In addition lowpass filtering of all control signals, both remote and software based is included and can be customised by adapting the parameters under {\ttfamily \hyperlink{setup__macros_8h_source}{firmware/setup\+\_\+macros.\+h}}. The main and control loop time is configured and tested at {\itshape 50\+Hz}. Theoretically much higher frequencies are possible, however interrupt reading capability greatly deteriorates with a fast main loop frequency. Therefore it is recommended to not vary these frequencies to higher values or otherwise stable performance can not be guaranteed. Also the firmware provides a led signaling class which provides easily accessible information about the current \hyperlink{namespaceSPMB}{S\+P\+MB}\textquotesingle{}s state machine status.


\begin{DoxyItemize}
\item {\bfseries Main loop} in \hyperlink{firmware_8ino_source}{firmware.\+ino} depending on
\begin{DoxyItemize}
\item {\bfseries Status\+Indicator}
\begin{DoxyItemize}
\item \hyperlink{si_8h_source}{si.\+h} -\/ L\+ED signaling and exposing information to environment
\end{DoxyItemize}
\item {\bfseries Interrupt\+Manager}
\begin{DoxyItemize}
\item \hyperlink{input__rc_8h_source}{input\+\_\+rc.\+h} -\/ Input via RC
\end{DoxyItemize}
\item {\bfseries R\+O\+S\+Interface}
\begin{DoxyItemize}
\item \hyperlink{ros__interface_8h_source}{ros\+\_\+interface.\+h} -\/ Input via R\+OS and publish actuated control
\end{DoxyItemize}
\item {\bfseries Low\+Pass}
\begin{DoxyItemize}
\item \hyperlink{lowpass_8h_source}{lowpass.\+h} -\/ Filter for output elements
\end{DoxyItemize}
\item {\bfseries Output\+Driver\+I2C}
\begin{DoxyItemize}
\item \hyperlink{output__i2c_8h_source}{output\+\_\+i2c.\+h} -\/ Output via I2C to timer P\+CB
\end{DoxyItemize}
\item {\bfseries Setup\+S\+P\+MB}
\begin{DoxyItemize}
\item \hyperlink{setup_8h_source}{setup.\+h} -\/ Initial pin configurations and setup of objects
\end{DoxyItemize}
\item {\bfseries State\+Machine}
\begin{DoxyItemize}
\item \hyperlink{sm_8h_source}{sm.\+h} -\/ Execute main logic loop and interface with all involved elements
\end{DoxyItemize}
\end{DoxyItemize}
\end{DoxyItemize}





\subsection*{Features}


\begin{DoxyItemize}
\item RC Forwarding
\item Safety logic for both R\+O\+S-\/ and R\+C-\/based control
\item Actuation output via R\+OS for debugging and learning of manual controlled maneuvers or continuation of actuation-\/based estimator while supervised intervention
\item Switching logic with state machine considering idle conditions, various error cases and providing supervisor interrupt via braking maneuver
\item {\itshape 50 Hz} actuation and lowpass filtered control
\item O\+OP based modules with independent use cases both with and without R\+OS usage (serial and ros nodehandle.\+loginfo debugging possible)
\item Installation of udev rules for unique device node under {\ttfamily /dev/spmb}
\item Scripts for installing ros libraries to Arduino library folder {\ttfamily $\sim$/\+Arduino/libraries/}
\item Doxygen based A\+PI as reference as basis for continuoous integration
\end{DoxyItemize}





\subsection*{Execute the R\+OS Interface}


\begin{DoxyItemize}
\item Run {\ttfamily roslaunch spmb run.\+launch}
\end{DoxyItemize}

({\itshape If any configuration parameters regarding baudrate have been adjusted then parse the corresponding parameter in the launch file e.\+g. {\ttfamily roslaunch spmb run.\+launch baud\+:=yourbaudrateinteger}})


\begin{DoxyItemize}
\item After approximately 3-\/4 seconds the terminal will output
\end{DoxyItemize}


\begin{DoxyCode}
1 setting /run\_id to 4e4bd514-a7b2-11e9-a7b3-4485007bad36
2 process[rosout-1]: started with pid [25122]
3 started core service [/rosout]
4 process[serial\_node-2]: started with pid [25128]
5 [INFO] [1563271988.137866]: ROS Serial Python Node
6 [INFO] [1563271988.149294]: Connecting to /dev/spmb at 57600 baud
7 [ERROR] [1563272005.372626]: Unable to sync with device; possible link problem or link software version
       mismatch such as hydro rosserial\_python with groovy Arduino
8 [INFO] [1563272005.404425]: Note: publish buffer size is 100 bytes
9 [INFO] [1563272005.406264]: Setup publisher on actuated [spmb/actuated]
10 [INFO] [1563272005.418630]: Note: subscribe buffer size is 100 bytes
11 [INFO] [1563272005.419878]: Setup subscriber on request [spmb/request]
\end{DoxyCode}


({\itshape Note that initially there will be a single error output regarding a sync problem. After this the publishers should be initialised and no further error messages should be shown.})





\subsection*{Dependencies}

\subsubsection*{Install Arduino Libraries}

Go to {\ttfamily cd $\sim$/\+Arduino/libraries}


\begin{DoxyItemize}
\item {\ttfamily git clone \href{https://github.com/adafruit/Adafruit-PWM-Servo-Driver-Library.git}{\tt https\+://github.\+com/adafruit/\+Adafruit-\/\+P\+W\+M-\/\+Servo-\/\+Driver-\/\+Library.\+git}}
\item {\ttfamily git clone \href{https://github.com/maniacbug/StandardCplusplus.git}{\tt https\+://github.\+com/maniacbug/\+Standard\+Cplusplus.\+git}}
\item Download and install Arduino (v1.\+8.\+5) (x64 \href{https://www.arduino.cc/download_handler.php?f=/arduino-1.8.5-linux64.tar.xz}{\tt https\+://www.\+arduino.\+cc/download\+\_\+handler.\+php?f=/arduino-\/1.\+8.\+5-\/linux64.\+tar.\+xz})
\item Build the repository to generate header files for messages with {\ttfamily catkin build} in any directory of your catkin workspace
\item Source your workspace with {\ttfamily source $\sim$/your\+\_\+workspace/devel/setup.bash}
\item Run the {\ttfamily make\+\_\+library.\+sh} script in the root of this repository
\item If desired install the udev rule file under {\ttfamily resources/99-\/spmb.\+rules}, which will install the permanent device node {\ttfamily dev/spmb} to avoid the deviating node identifier tty\+A\+C\+Mx with x changing based on the amount of tty\+A\+CM devices or reconnections of the same device)
\end{DoxyItemize}

\subsubsection*{Interface with R\+OS}


\begin{DoxyItemize}
\item After installing the udev rules (see below) the R\+OS Interface can be accessed with {\ttfamily rosrun rosserial\+\_\+python serial\+\_\+node.\+py \+\_\+port\+:=/dev/spmb \+\_\+baud\+:=57600}
\item The {\ttfamily rosrun} command is conveniently embedded into the launch file under {\ttfamily launch/run.\+launch} (see {\itshape Execute the R\+OS Interface} section above)
\end{DoxyItemize}

\subsubsection*{Install udev-\/rules}


\begin{DoxyItemize}
\item You can find the rules under the folder {\ttfamily resources} with the name {\ttfamily 99-\/spmb.\+rules}
\item Easiest way to identify the device node is to unplug all usb devices from your work station except the Arduino U\+NO. Then before connecting the Arduino you can list the devices with {\ttfamily ls /dev} and then after connecting the Arduino entering the command {\ttfamily ls /dev} should show a new device node entry. (Usually {\ttfamily /dev/tty\+A\+C\+Mx} where x is some number relating to the amount of currently registered tty\+A\+CM devices)
\item Assuming that your Arduino identifies with {\ttfamily /dev/tty\+A\+C\+M0} you can list the hardware information with the command
\end{DoxyItemize}


\begin{DoxyCode}
1 udevadm info -ap /sys/class/tty/ttyACM0
\end{DoxyCode}



\begin{DoxyItemize}
\item From there extract the serial attribute with
\end{DoxyItemize}


\begin{DoxyCode}
1 udevadm info -ap /sys/class/tty/ttyACM0 | grep "^    ATTRS\{serial\}=="
\end{DoxyCode}


which should show something similar to


\begin{DoxyCode}
1 $ udevadm info -ap /sys/class/tty/ttyACM0 | grep "^    ATTRS\{serial\}=="
2     ATTRS\{serial\}=="85430353331351E0D120"
3     ATTRS\{serial\}=="0000:00:14.0"
\end{DoxyCode}



\begin{DoxyItemize}
\item Open the file {\ttfamily 99-\/spmb.\+rules} under {\ttfamily resources/} and replace the default entry for serial {\ttfamily A\+T\+T\+RS\{serial\}==\char`\"{}55736313238351219281\char`\"{}} with your terminal output e.\+g. {\ttfamily A\+T\+T\+RS\{serial\}==\char`\"{}85430353331351\+E0\+D120\char`\"{}}
\item Copy the file into the udev folder using super user privileges
\end{DoxyItemize}


\begin{DoxyCode}
1 sudo cp resources/99-spmb.rules /etc/udev/rules.d/
\end{DoxyCode}



\begin{DoxyItemize}
\item Reload and trigger the rules using
\end{DoxyItemize}


\begin{DoxyCode}
1 udevadm control --reload-rules && udevadm trigger
\end{DoxyCode}


{\itshape If {\ttfamily ls /dev} doesn\textquotesingle{}t show the entry} {\bfseries dev/spmb} {\itshape try rebooting your work station and if the entry still doesn\textquotesingle{}t show reiterate throught the instructions to make sure you followed them correctly.}





\subsection*{Contribution}

Pull requests from forked repositories and opening new issues are very welcome and much appreciated.

Contributions are invited to adhere to the \href{https://google.github.io/styleguide/cppguide.html}{\tt C++ Google Style Guide}.

{\bfseries Maintainer\+:} Philipp Rothenhäusler ({\itshape phirot a t kth.\+se $\vert$ philipp.\+rothenhaeusler a t gmail.\+com}) 